\documentclass[a4paper,12pt,twoside]{article}
% \usepackage[paperwidth=227pt,paperheight=190pt]{geometry}
% \special{papersize=227pt,190pt}
\usepackage[usenames,dvipsnames,svgnames,table]{xcolor}
\definecolor{light-gray}{gray}{0.72}
\usepackage[absolute,overlay]{textpos}
\usepackage{times}
\usepackage{verbatim}
\usepackage{tikz}
\usetikzlibrary{arrows,shapes,positioning,backgrounds,fit}

\tikzset{
  % Define standard arrow tip
  >=stealth',
  % Define style for boxes
  elli/.style={
    ellipse,
    draw=black,
    text centered
  },
  recti/.style={
    rectangle,
    draw=black,
    text centered
  },
  punkt/.style={
    rectangle,
    rounded corners,
    draw=black, very thick,
    %maximum width=6em,
    %text width=6.5em,
    minimum height=2em,
    text centered},
  punkk/.style={
    rectangle,
    rounded corners=3mm,
    draw=black, very thick,
    %text width=6.5em,
    minimum height=2em,
    text centered},
  punkr/.style={
    rectangle,
    draw=black, very thick,
    %maximum width=6em,
    %text width=6.5em,
    minimum height=2em,
    text centered},
  punkr2/.style={
    rectangle,
    draw=black, very thick,
    inner sep = 4pt,
    %maximum width=6em,
    %text width=6.5em,
    %minimum height=2em,
    text centered},
 punkk2/.style={
    rectangle,
    rounded corners=3mm,
    draw=black, very thick,
    inner sep = 4pt,
    %text width=6.5em,
    %minimum height=2em,
    text centered},
  % Define arrow style
  pil/.style={
    ->,
    thick,
    shorten <=2pt,
    shorten >=2pt,}
}

\pagestyle{empty}
\begin{document}

\usetikzlibrary{arrows,shapes,positioning,backgrounds,fit}

\tikzset{
  % Define standard arrow tip
  >=stealth',
  % Define style for boxes
  elli/.style={
    ellipse,
    draw=black,
    text centered
  },
  recti/.style={
    rectangle,
    draw=black,
    text centered
  },
circ2/.style={
    circle,
    draw=black,
    text centered,very thick
  },
  punkt/.style={
    rectangle,
    rounded corners,
    draw=black, very thick,
    %maximum width=6em,
    %text width=6.5em,
    minimum height=2em,
    text centered},
  punkk/.style={
    rectangle,
    rounded corners=3mm,
    draw=black, very thick,
    %text width=6.5em,
    minimum height=2em,
    text centered},
  punkr/.style={
    rectangle,
    draw=black, very thick,
    %maximum width=6em,
    %text width=6.5em,
    minimum height=2em,
    text centered},
  punkr2/.style={
    rectangle,
    draw=black, very thick,
    inner sep = 4pt,
    %maximum width=6em,
    %text width=6.5em,
    %minimum height=2em,
    text centered},
 punkk2/.style={
    rectangle,
    rounded corners=3mm,
    draw=black, very thick,
    inner sep = 4pt,
    %text width=6.5em,
    %minimum height=2em,
    text centered},
  % Define arrow style
  pil/.style={
    ->,
    thick,
    shorten <=2pt,
    shorten >=2pt,}
}

\begin{tikzpicture}[node distance=1cm, auto,]
\path[use as bounding box] (-6.5,.2);
% \node (txt) {txt file};
% \node[right=of txt, text width=3cm] (df) {data frame (RAM object)};
% \node[right=of df] (rd) {RData};
% \node[recti,left=of txt, yshift=-5mm] (cf) {\texttt{cutFile}};
% \node[below=of cf] (dir) {directory};



% \node (rd) {RData};
% \node[right=of rd, text width=3cm, xshift=-2mm] (df) {data frame (RAM object)};
% \node[right=of df,xshift=-3mm] (txt) {txt file};
% \node[recti,right=of txt, yshift=-5mm] (cf) {\texttt{cutFile}};
% \node[below=of cf] (dir) {directory};


\node[recti] (afe) {\texttt{allFitnessEffects}};
\node[below=of afe, text width=2.8cm] (afeo) {Effects of genes on fitness (\textit{list})};
\node[recti, right=of afe] (ame) {\texttt{allMutatorEffects}};
\node[below=of ame, text width=3.3cm] (ameo) {Effects of genes on mutation (\textit{list})};
\draw[->] (afe) to (afeo);
\draw[->] (ame) to (ameo);



\node[above=of afe, text width=3.2 cm, yshift=3mm,xshift=7mm] (dag) {Order restrictions DAG
  (\textit{data frame})};
\node[above=of afe, text width=1.5 cm, xshift=-18mm] (oe) {Order effects (\textit{vector})};
\node[above=of afe, text width=1.5 cm, yshift =4mm, xshift=35mm] (epi) {Epistasis (\textit{vector})};
\node[above=of afe, text width=2.9 cm, yshift=4mm, xshift=65mm] (nig) {Non-interacting genes (\textit{vector})};
\node[recti, above=of dag,yshift=0mm] (simo) {\texttt{simOGraph}};

\draw[->] (oe) to (afe);
\draw[->] (epi) to (afe);
\draw[->] (nig) to (afe);
\draw[->] (dag) to (afe);
\draw[->] (simo) to (dag);

\draw[->] (epi) to (ame);
\draw[->] (nig) to (ame);
\node[above=of afe, text width=1.7 cm, yshift=-5mm,xshift=-39mm] (rfl) {Random fitness landscape (\textit{matrix})};
\node[recti,above=of rfl, text width=2.0 cm,yshift=-1.2mm] (rfi) {\texttt{rfitness}};
\draw[->] (rfi) to (rfl);
\draw[->] (rfl) to (afe);

% \node[recti,below=of afe, xshift=-5mm] (pfe) {\texttt{plot (plot.fitnessEffects}};
% \node[below=of pfl] (ppfe) {\textit{DAGPlot of fitness landscape}};

\node[recti,below= of afeo, text width = 3.9cm, yshift = 1mm, xshift=15mm] (oii)
{
  \texttt{oncoSimulIndiv}
  \texttt{oncoSimulPop}
  \texttt{oncoSimulSample}
};

\draw[->] (afeo) to (oii);
\draw[->] (ameo) to (oii);


\node[circ2,below=of oii, text width=2.9cm] (str) {Simulated trajectories (\textit{lists})};
\node[recti,left=of str, xshift=-5mm] (ppio) {
 \texttt{plot}};

\node[recti,left=of str, text width=3.94cm, yshift = -30mm, xshift=5mm]
(sampp) {
  \texttt{samplePop}
  \texttt{sampledGenotypes}
};
\node[below=of sampp, text width=3.9cm] (samppout) {Genotypes, genotype
  frequencies, diversity (\textit{matrix, data frame})};

\draw[->] (oii) to (str);
\draw[->] (str) to (ppio);
\draw[->] (str) to (sampp);
\draw[->] (sampp) to (samppout);


\node[recti,below=of str, xshift=0.5mm, yshift=-0mm,text width=3.99cm] (lod) {
  \texttt{LOD},
  \texttt{POM}
  \texttt{diversityLOD/POM}
%  \texttt{diversityPOM}
};
\node[below=of lod, text width=3.99cm] (lodout) {
  Evolut.\ predictability (\textit{vector, scalars})};

\node[recti,right=of str, xshift=0mm, yshift = -20mm, text width=3.7cm] (pphyl) {  %% (pphyl)
  \texttt{plotClonePhylog}
};

\draw[->] (str) to (lod);
\draw[->] (lod) to (lodout);
\draw[->] (str) to (pphyl);






\node[recti,below=of rfl, yshift = -1mm, xshift=0mm] (pfl) {\texttt{plot}};

\draw[->] (rfl) to (pfl);
% %%% To remove for Bioinfo
% %\node[recti,below=of rfl, yshift = -9mm, xshift=-12mm] (pfl) {\texttt{plotFitnessLandscape}};
% 
\node[below=of pfl, xshift=-15mm,yshift=5mm] (ppfl) {(\textit{Plot of fitness landscape})};
\draw[->] (pfl) to (ppfl);
\node[recti,left=of rfl] (tom) {\texttt{to\_Magellan}};
\node[below=of tom] (tfm) {(\textit{text file})};
\draw[->] (tom) to (tfm);
\draw[->] (rfl) to (tom);

\node[recti,below= of afeo, text width = 5.2cm, yshift = -1mm, xshift=75mm] (eafe)
{
  \texttt{evalAllFitnessEffects}
  \texttt{evalAllMutatorEffects}
};

\node[below=of eafe, text width=4.5cm, yshift=2mm] (eafeout) {
  Fitness/mutation of all genotypes (\textit{data frame})};

\draw[->] (afeo) to (eafe);
\draw[->] (ameo) to (eafe);
\draw[->] (eafe) to (eafeout);

\node[below=of pphyl, xshift=0mm, text width=3cm] (pphylout) {
(\textit{Plot of  genealogies/phylogenies})};
\draw[->] (pphyl) to (pphylout);

\node[recti,below=of eafeout, yshift = 6mm, xshift=0mm] (eafeoutpl) {\texttt{plot}};
\node[below=of eafeoutpl, xshift=0mm,yshift=6mm] (pleafeout) {(\textit{Plot of fitness landscape})};
\draw[->] (eafeout) to (eafeoutpl);
\draw[->] (eafeoutpl) to (pleafeout);


\node[left=of ppio, xshift=2mm,yshift=0mm,  text width=4.2cm] (outppio) {(\textit{Plot of
    genotypes abundances over time})};
\draw[->] (ppio) to (outppio);

% \node[left=of simo, xshift = 4mm] (hidden) {};
% \draw[->] (simo) to (hidden);

\end{tikzpicture}



%%% Overlay a couple of figures


% \begin{textblock*}{40mm}(137mm,74mm)
% \includegraphics[%
%   width=4.3cm,
%   keepaspectratio]{overlay-f1.pdf}
% \end{textblock*}



% %% pdfcrop overlay-f2.pdf
% \begin{textblock*}{70mm}(132mm,155mm)
% \includegraphics[%
%   width=4.3cm,
%   keepaspectratio]{overlay-f2-crop.pdf}
% \end{textblock*}


% \begin{textblock*}{70mm}(43mm,65mm)
% \includegraphics[%
%   width=3.1cm,
%   keepaspectratio]{overlay-f3-crop.pdf}
% \end{textblock*}


% \begin{textblock*}{70mm}(73mm,3mm)
% \includegraphics[%
%   width=1.3cm,
%   keepaspectratio]{overlay-f4-crop.pdf}
% \end{textblock*}

\end{document}


%%% Local Variables:
%%% mode: latex
%%% TeX-master: "figure1"
%%% End:



%% I create a png. Either via convert or from ktikz. Similar final result.
%% If using convert do something like:
%% convert -density 200 -trim figure1-crop.pdf -quality 100 -sharpen 0x1.0 -flatten figure1.png
